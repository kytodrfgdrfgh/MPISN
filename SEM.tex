\documentclass[14pt,a4paper,oneside]{extarticle}
\usepackage{amsmath}\usepackage{amsfonts}\usepackage{amssymb}\usepackage{pgfplots}
\usepackage{setspace}\usepackage{mathtools}\usepackage{tikz}\usetikzlibrary{arrows}
\usepackage[left=3cm,right=1cm, top=2cm,bottom=2cm,bindingoffset=0cm]{geometry}
\usepackage[T2A]{fontenc}
\usepackage[utf8]{inputenc}
\usepackage[russian,english]{babel}
\usepackage{enumitem}
\usepackage{ragged2e}
\makeatletter\AddEnumerateCounter{\asbuk}{\russian@alph}{щ}\makeatother
\DeclareMathOperator{\sinc}{sinc}\pgfplotsset{compat=1.18}\usepackage{setspace}
% \onehalfspacing
\newcommand{\tb}[1]{\textbf{#1}}
\newcommand{\pict}{\[\tb{Картинка}\]}
\newcommand{\fo}{\[\tb{Формула}\]}
\usepackage{tikz}
\usetikzlibrary{shapes.geometric,shapes.arrows,decorations.pathmorphing}
\usetikzlibrary{matrix,chains,scopes,positioning,arrows,fit, automata}
\usepackage{bookmark}
\usepackage{blindtext}
% \setcounter{secnumdepth}{0}
\usepackage{graphicx}
\graphicspath{ {./imgs/} }
\usepackage{relsize}
\usepackage{indentfirst}
\usepackage{float}
\usepackage{makecell}
% \usepackage{sectsty}

% \sectionfont{\large}
% \sectionfont{\normalsize}
% \subsectionfont{\normalsize}
% \paragraphfont{\normalsize}

\usepackage{amsthm}
\newtheorem*{theorem}{Теорема}
% \makeatletter
% \renewcommand{\@seccntformat}[1]{\csname the#1\endcsname.\quad}
% \makeatother
\hypersetup{%
    pdfborder = {0 0 0}
}

\usepackage{ragged2e}

% \setlength{\RaggedRightParindent}{\parindent}

\newcommand{\pic}[3]{
	\begin{figure}[#1]
		\begin{center}
			\includegraphics[width=#2]{imgs/#3}
		\end{center}
		% \vspace{-5mm}\caption{#4}\label{fig:#5}
	\end{figure}
}

%\newlength{\normalparindent}
%\setlength{\normalparindent}{\parindent}
%\raggedright
%\setlength{\parindent}{\normalparindent}
\sloppy

\usepackage{tocloft}
\renewcommand{\cftsecdotsep}{\cftdotsep}

\begin{document}

%\justifying

% \RaggedRight

% \emergencystretch 3em

\begin{titlepage}
    \begin{center}
        \vspace*{1cm}

        \Huge
        \textbf{Методы прецизионных измерений в спутниковой навигации}
        \vspace{1.5cm}

        \vfill
        \Huge
        \textbf{Экзамен}\\
        \vspace{0.5cm}
        \LARGE
        Преподаватель:\\Дмитрий Вячеславович Семенов

        \vspace{1.5cm}

        \vfill

        \LARGE
        РЛ1-112

        \vspace{0.8cm}

        % \includegraphics[width=0.4\textwidth]{university}

        \Large
        МГТУ им. Н.Э. Баумана

    \end{center}
\end{titlepage}

\addtocontents{toc}{\protect\sloppy}
\tableofcontents

\section{Обзор современных СРНС. Основные источники погрешностей НВО.}

\pic{H}{\textwidth}{1}

Основные источники погрешностей:

\begin{itemize}
    \item Ошибки эфемеридно-временного обеспечения
    \item Ионосфера
    \item Тропосфера
    \item Многолучевость
    \item Шумы и задержки в аппаратуре потребителя
    \item Шумы и задержки на борту НКА
    \item Прочие задержки и помехи
\end{itemize}

Сумма этих ошибок может быть вычислена в любой точке земного шара, если с высокой точностью известны координаты приемника, но она действительна только для незначительной области в течение небольшого периода времени.

Для этого созданы различные системы дифференциальной коррекции. В этом случае используется приёмник расположенный в точке с известными координатами (базовая станция), а второй приёмник собирает данные в точках с неизвестными координатами (передвижной приёмник).

Данные, полученные в точке с известными координатами, используются для определения ошибок содержащихся в спутниковом сигнале. Затем информация с базовой станции совместно обрабатывается с данными передвижного приёмника, вместе с учётом ошибок содержащихся в спутниковом сигнале, что позволяет устранить ошибки в координатах полученных на передвижном приёмнике.

\section{Пространственная и временная корреляция погрешностей НВО на примере погрешности эфемеридного обеспечения.}

В 90е годы США попытались ограничить точность НВО за счет преднамеренного загрубления передаваемой в составе НС ЭВИ, однако это не принесло желаемого результата, поскольку потребители начали применять методы дифференциальной навигации, основанные на совместной разностной обработке результатов независимых измерений, полученных в пространственно разнесенных точках приема.

Принципиальная возможность осуществления дифференциальной коррекции следует из 2 экспериментально подтвержденных фактов:

\begin{itemize}
    \item значение суммарной ЭПД, обусловленной погрешностями ЭВИ и трассой распространения сигналов, можно оценить с помощью АП, установленной в точке с известными координатами (базовая станция).
    \item Погрешности измерений, выполненных по одному и тому же НКА потребителями, находящимися в различных точках пространства, содержат пространственно коррелированные составляющие, которые могут быть скомпенсированы методами дифференциальной обработки этих измерений.
\end{itemize}

В основе дифференциальной навигации лежит свойство пространственно-временной коррелированности погрешностей измерений радионавигационных параметров сигналов НС, проведенных в различных точках пространства в близкие моменты времени.
Радиус пространственной корреляции – область значений в заданном пространстве, в которой параметры коррелированны.

Анализ баланса погрешностей НВО дает основания утверждать, что практически для любого потребителя некореллированными(относительно других потребителей) можно считать погрешности собственннойАП, а также погрешности, обусловленные многолучевым распространением сигналов вблизи точки расположения антенны.

В то же время для всех потребителей, наблюдающих данный НКА системные погрешности, вносимые бортовой аппаратурой НКА, а также данными ЭВИ, «закладываемыми НКУ», имеют практически 100\% корреляцию.

Кроме того для значительных областей на земной поверхности и вблизи нее в той или иной степени коррелированными являются погрешности, возникающие на трассе распространения сигнала от НКА до потребителя в ионосфере и тропосфере.

Т.о. существует объективная возможность повысить точность НВО ГНСС за счет совместной разностной обработки измерений, полученных разными потребителями, поскольку при этом снижается влияние коррелированных составляющих погрешностей.

Однако необходимо помнить, что использование разностных алгоритмов приводит к возрастанию дисперсии некоррелированной (шумовой) составляющей погрешности, поскольку дисперсия разности независимых измерений равна сумме их дисперсий. Поэтому АП, используемая в режиме дифференциальной навигации, должна иметь минимальный уровень флуктуационныхпогрешностей.

\pic{H}{\textwidth}{7}

Для оценки зависимости ЭПД, обусловленной погрешностями эфемеридной информации и от расстояния между пользователем и БС, используем следующие выражения.

\[R_{\text{п}}^{\prime2}=R_{\text{БС}}^{\prime2}+d^2-2dR_{\text{БС}}^{\prime}\cos(\varphi-\alpha);\]
\[R_{\text{п}} ^ 2 = R_{\text{БС}} ^ 2 + d ^ 2 - 2 d R_{\text{БС}} \cos \varphi\]

\pic{H}{\textwidth/2}{2}

Точка со штрихом – расчетное местоположение НКА. Без штриха, фактическое.

Для НКА, расположенного под углами места более 5 градусов, при расстоянии между БС и потребителем d = 1000 км ЭПД $e_\text{ЭИ}$, обусловленную погрешности ЭИ, можно записать в следующем виде:

\[|\delta R_\text{БС}-\delta R_\text{{п}}|\leqslant\alpha d\sin\varphi=\frac{e_\text{ЭИ}}{R_\text{БС}}d\sin^2\varphi.\]

Т.о., ЭПД линейно возрастает с увеличением расстояния d, например, для d = 100 км и при погрешности эфемерид $e_\text{ЭИ}$ = 5м, некомпенсированная ЭПД $\delta R$ составит 0,025м.

\section{Пространственная и временная корреляция погрешностей НВО на примере тропосферной погрешности.}

Значение ЭПД $\delta R_{\text{тpoп}}$, обусловленное неполной компенсацией  влияния Тропосферной рефракции, можно определить по формуле

\[|\delta R_{\text{троп.п}}-\delta R_{\text{троп.пC}}|\leqslant d\frac{1}{H\cos\varphi}(1,4588+0,0029611N_{s})-\]
\[-\underbrace{0,3048[0,00586(N_s-360)^2+294](\varphi_\text{п}^{-2,30}-\varphi_\text{БС}^{-2,30})}_{x},\]

Где:

NS – индекс рефракции, среднее значение NS = 360; 

X – параметр, определяющий возрастание погрешности при малых (менее 10°) углах места НКА по отношению к потребителю.

Т.о. на расстоянии 100 км. Измерения по НКА, расположенном под углом места 45°, будут иметь расхождения тропосферных погрешностей на уровне 0,02 м.

\pic{H}{\textwidth/2}{3}

\section{Пространственная и временная корреляция погрешностей НВО на примере ионосферной погрешности.}

\pic{H}{\textwidth}{4}

\[\left|\delta R_{\text{ион.п}}-\delta R_{\text{ион.БС}}\right|=\left|\frac{1}{\text{соs }\Psi}-\frac{1}{\cos\Psi_{\text{n}}}\right|\frac{40,3}{f^{2}}N_{\Sigma}=\]
\[=\frac{d}{R_{\text{БС}}}\left|\frac{d}{R_{\text{БС}}}-\cos\Psi\right|\frac{40,3}{f^{2}}N_{\Sigma}.\]

Т.о. при типичном ПЭС $\mathbf{N}_{\Sigma}=50\mathsf{ TECU}$ ($50\cdot10^{16}$ электрон/м2)и расстоянии между П и БС 100 км некомпенсируемая ЭПД первичного измерения псевдодальностипо НКА, расположенному под углом места $\phi=45^{\circ}$, будет иметь значение 0,03м.

\pic{H}{\textwidth}{5}

\section{Погрешности эфемеридно-временного обеспечения.}

\begin{itemize}
    \item Шкала времени – временная «координатная» система, cвязанная с конкретными часами.
    \item Шкала времени системы (ШВС) –шкала времени, связанная с часами радионавигационной системы.
    \item Шкала времени потребителя (ШВП) –шкала времени,  связанная с часами потребителя.
    \item Бортовая шкала времени (БШВ)  – шкала времени, связанная с часами бортового синхронизирующего устройства (БСУ) конкретного НКА
    \item Шкала времени навигационного сигнала (ШВ\{НС\}) –шкала времени, связанная с ПСП дальномерного кода сигнала \{НС\} в фазовом центре антенны, излучающей этот сигнал
\end{itemize}

Погрешности обусловлены уходами бортового эталона, а также конечной точностью процедур сверки бортовой ШВ с системной. Как следствие, возникает смещение фаз излучаемых дальномерных кодов и меток времени, что приводит к погрешностям измерения РНП и в итоге ––к снижению точности НВО.

\[\delta S_{\text{БШB}}=c\delta T_{\text{БШB}}=c\left(t_\text{факт}-t_\text{{пр}}\right).\]

Функциональная зависимость ЭПД потребителя от рассогласования между фактическим отсчетом БШВ и прогнозным значением БШВ, рассчитанным потребителем с помощью соответствующей математической модели,транслируемой ему в составе НС.

Например, при расхождении реальных показаний БШВ и прогноза на 10 нс, ошибка измерения псевдодальности составит порядка 3 м для всех потребителей.

Ошибки определения времени за счет погрешности космического сегмента не учитывают погрешности распространения в среде и погрешности НАП. Они фактически определяют потенциальную точность потребителей, оснащенных двухчастотной НАП без учета внутренних задержек НАП.

В тех случаях, когда уход БШВ относительно СШВ превышает допустимые значения, проводится её коррекция либо путём фазирования (совмещения временных интервалов БШВ и СШВ) на физическом уровне, либо путём изменения цифрового кода БШВ на целое число единиц времени. Коррекция БШВ ГЛОНАСС производится таким образом, чтобы её сдвиг относительно системной ШВ не превосходил 10 нс, т. е. примерно соответствовал суточной нестабильности БШВ, которая для НКА «Глонасс-М» составляет $10^{-13}$.

Однако проведение таких непосредственных коррекций возможно только эпизодически, в моменты нахождения НКА в зоне видимости наземных средств наблюдения. В то же время максимальное отклонение БШВ относительно СШВ уже через 2 ч после коррекции составляет примерно 10 нс, а через сутки после коррекции достигает примерно 25нс для цезиевых и 110 нс для рубидиевых эталонов.

Поэтому наряду с непосредственной используется алгоритмическая коррекция, основанная на прогнозировании систематической составляющей ухода БШВ. Возможность такой коррекции также базируется на результатах многолетних наблюдений, которые показывают, что на интервалах времени порядка суток систематическая составляющая ухода БШВ изменяется по закону, близкому к линейному.

С учётом сказанного, для данного НКА на каждом витке рассчитываются частотно-временные поправки, которые дважды в сутки закладываются в бортовой компьютер НКА, а затем включаются в НС и передаются потребителю. Поправки представляют собой два параметра линейной модели смещения БШВ относительно СШВ. Эта модель заложена в программном обеспечении АП, что позволяет учитывать принятые поправки при НВО. Для НКА «Глонасс-М» с цезиевым эталоном частоты при такой коррекции погрешность прогноза ухода БШВ на 12 ч составляет в среднем 7,2 нс (ЭПД = 2,1 м). Для НКА GPS всех модификаций погрешность частотно-временного обеспечения составляет 4,4 нс (ЭПД = 1,2 м).

Существует ещё одна составляющая погрешности времени – фазовый сдвиг (групповая задержка) навигационного сигнала в аппаратуре при его распространении от бортового эталона до передающей антенны НКА. Систематическая составляющей этой задержки учитывается при расчете параметров коррекции БШВ, передаваемых потребителю в составе НИ. Случайная составляющая групповой задержки, не превышающая 3 нс (ЭПД = 1,0 м), входит в дальномерную погрешность.

Погрешности $\delta R$ обусловлены неточностью расчёта параметров орбит НКА в НКУ и непрогнозируемыми отклонениями реальной орбиты НКА относительно экстраполированной. Эфемеридная погрешность $\delta R$ имеет три компоненты: продольную $\delta_{l}$ (по касательной к траектории), радиальную $\delta_{r}$ (по направлению к центру Земли) и нормальную $\delta_{n}$ (перпендикулярную плоскости орбиты). Соответствующая эфемередной погрешности составляющая ЭПД $\delta_{\text{э}}$ представляет собой проекцию вектора {$\delta_{l}$; $\delta_{r}$; $\delta_{n}$} на линию, соединяющую потребителя и НКА.

\section{Тропосферная погрешность. Модель Хопфилда.}

На скорость распространения сигнала в тропосфере оказывают влияние нейтральные атомы и молекулы, различие концентрации которых приводит к появлению локальных неоднородностей, отличающихся значением диэлектрической проницаемости и, соответственно, значением тропосферной задержки.

Поскольку на частотах менее 15 ГГц значение задержки в тропосфере не зависит от частоты (дисперсия сигнала отсутствует), измерить эту задержку с помощью двухчастотного сигнала невозможно. Поэтому тропосферная задержка оценивается прогнозным методом. Наибольшее распространение получили модели Хопфилд (предложена в 1969 г.) и Саастамоинена (предложена в 1973 г.).

В модели Хопфилд на основе измерений атмосферного давления p, температуры T и парциального давления водяных паров e определяют приращение псевдодальности $\delta R_{\text{троп}}$, обусловленное тропосферной задержкой:

\[\begin{gathered}
        \delta R_{\text{троп}}=\delta R_{\text{сух}}+\delta R_{\text{вл}}; \\
        {\delta}R_{\text{вл}}=\frac{10^{-6}}{5}\frac{(-12,96\cdot T+3,718\cdot10^{5})}{\sin\sqrt{E^{2}+2,25}}\frac{e}{T^{2}}h_{\text{вл}}; \\
        \delta R_{\text{cyx}}=\frac{10^{-6}}{5}\frac{77,64}{\sin\sqrt{E^{2}+2,25}}\frac{p}{T}h_{\text{cyx}}.
    \end{gathered}\]

    \[h_{\mathrm{cyx}}=40136+148,72\cdot(T-273,16)\]

Где:

$E$ - угол места направления АП – НКА относительно плоскости горизонта, рад;

$h_\text{вл}$ - высота влажного слоя тропосферы (в зависимости от региона Земного шара изменяется от 10 000 до 13 000 м, на практике принимают 11 500 м)

\section{Тропосферная погрешность. Модель Саастомоиена.}

Модель Саастамоинена позволяет непосредственно оценить $\delta R_{\text{троп}}$:

\[\delta R_{\text{троп}}=\frac{0,002277}{\cos\varphi}\left[p+\left(\frac{1255}{T}+0,05\right)e-\mathrm{tg}^{2}\varphi\right]\]

В обеих моделях давления p и e измеряются в миллибарах, а температуру T – в градусах Кельвина.

Модифицированная модель Саастамоинена учитывает изменение значения задержки в зависимости от высоты АП и зенитного угла:

\[\delta R_{\text{троп}}=\frac{0,002277}{\cos\varphi}\left[p+\left(\frac{1255}{T}+0,05\right)e-B\mathrm{tg}^{2}\varphi\right]+D\]

где B – коэффициент, учитывающий высоту АП, мбар; D – коэффициент, учитывающий угол места НКА и высоту АП, м.

Как показывают расчёты, величина тропосферной задержки навигационного радиосигнала НКА для потребителя, находящегося на уровне моря, при условиях отсутствия осадков (p = 1013,25 мбар, T=273,16 K, e = 0 мбар) может составлять от 2,3 м (ЭПД) для спутников, расположенных вблизи зенита, до 25 м (ЭПД) для спутников, расположенных под углами места менее 5$^\circ$. С учётом этого обстоятельства в рабочее созвездие, по которому проводятся НВО, рекомендуется включать только те НКА, угол возвышения которых превышает некоторое значение (угол маски), устанавливаемое потребителем. Обычно этот угол составляет 5...10$^\circ$.

Хотя значение тропосферной задержки при постоянном угле места медленно меняется во времени, его вклад в общий бюджет погрешности может изменяться гораздо быстрее из-за изменения угла места НКА вследствие его движения по небесной сфере.

\section{Ионосферная погрешность. Общая и упрощенная модель вычисления приращения псевдодальности вследствие ионосферной задержки.}

Ионосфера, в общем значении — это слой атмосферы планеты, сильно ионизированный вследствие облучения космическими лучами,
представляет собой природное образование разреженной слабоионизованной плазмы.

Это часть верхней атмосферы, где плотность свободных электронов достаточна, чтобы оказывать значительное влияние на распространение радиоволн.

Ионосфера является диспергирующей средой, т. е. групповая и фазовая скорость распространения в ней сигнала зависят от частоты последнего, что обусловлено высокой концентрацией свободных электронов $N_e$, которая превышает $10^3$ эл/см3.

Различия в диэлектрической проницаемости слоёв, расположенных на разных высотах, а также наличие локальных неоднородностей, приводит к тому, что задержка сигнала НКА в ионосфере (по сравнению со случаем его распространения в вакууме) изменяется в широких пределах в зависимости от района Земли, где находится потребитель, времени суток, года, солнечной и геомагнитной активности и т. д.

Общее приращение псевдодальности $\delta R_{\text{ион}}$, обусловленное ионосферой, может быть определено следующим выражением:

\[\delta R_\text{ион}=\frac{40,3}{f^2}\intop_{L_\text{н}} ^ { L _\text{в}} N _ e ( l ) d l\]

На практике для упрощения вычисления ионосферной задержки модель заменяют на более простую модель, в которой ионосфера рассматривается как материальная точка (подионосферная точка) P, расположенная на высоте $h_\text{ИТ}$ = 350 км:

\[\delta R_{\text{ион}}=c\delta\tau_{\text{ион}}=\frac{1}{\cos\Psi}\frac{40,3}{f^{2}}N_{\Sigma};\quad\sin\Psi=\frac{R_{3}}{R_{3}+h_{\text{ИT}}}\sin\varphi.\]

\pic{H}{\textwidth/2}{6}

\section{Ионосферная погрешность. Вычисление ионосферной задержки с помощью модели Клобушара.}

Однако в несложной коммерческой АП чаще используется компенсация данной погрешности на основе модельного прогноза состояния ионосферы. В 1986 г. была предложена модель Клобушара, а в 2001 г. Лаборатория аэрономии и распространения радиоволн Международного центра теоретической физики в Триесте (Италия) и Институт геофизики, астрофизики и метеорологии Университета г. Грац (Австрия) разработали модель NeQuick.

Исходными данными для расчёта являются восемь параметров модели $[\alpha_1,\alpha_2,\alpha_3,\alpha_4]$ и $[\beta_1,\beta_2,\beta_3,\beta_4]$, которые передаются в составе навигационного сообщения НКА GPS, географические широта B и долгота L АП, азимут Az и угол места El линии визирования НКА из этой точки.

Для определения с помощью модели Клобушара ионосферной задержки $\delta\tau_{\text{ион}}$ относящейся к моменту времени T, используют следующий алгоритм:

\begin{enumerate}
    \item Рассчитывают $\Psi=0,0137/(El+0,11)-0,022$ – центральный Земной угол между векторами, направленными из центра Земли на АП и НКА.
    \item Рассчитывают широту точки «прокола» ионосферы (IPP – ionospheric pierce point) \[B_\text{ИT}=B+\Psi\cos Az\]. Если $\text{B}_\text{ит}{ > + 0 , 4 1 6 , \text{B}_\text{ит}{ = 0 , 4 1 6 }}.$ Если $\text{B}_\text{ит}{ < - 0 , 4 1 6 , \text{B}_\text{ит}{ = -0 , 4 1 6 }}.$
    \item Рассчитывают долготу IPP \[L_{\text{ИТ}}=L+\Psi\frac{\sin Az}{\cos B_{\text{ИТ}}},\quad\text{если}\quad|B_{\text{ИТ}}|\leqslant0,416;\]
    \item Рассчитывают геомагнитную широту IPP \[\varphi_{\text{ИT}}=B_{\text{ИT}}+0{,}064\cdot\cos(L_{\text{ИT}}-1{,}617);\]
    \item Рассчитывают локальное время в IPP \[t=43200\cdot L_\text{ИТ}+T.\] Если $t\geq 86400$ то $t=t-86400$; если $t<0$, то $t=t+86400$
    \item Рассчитывают амплитудные составляющие задержки \[\begin{aligned}
                   & A_{1} =5\cdot10^{-9}\mathrm{c};                                                                                                                         \\
                   & A_{2} =\begin{cases}\sum\limits_{i=1}^4\alpha_i\varphi_{\text{ИТ}}^i,&\text{если }A_2\geqslant0;\\0,&\text{если }A_2<0\text{ c}\end{cases}
              \end{aligned}\]
    \item Рассчитывают период и фазовые составляющие \[\begin{aligned}
                   & A_{3} =\frac{2\pi(t-50400)}{A_{4}}\text{paд};                                                                                                                     \\
                   & A_{4} =\begin{cases}\sum\limits_{i=1}^4\beta_i\varphi_{\text{ИТ}}^i,&\text{ecли }A_4\geqslant72000;\\72000,&\text{ecли }A_4<7200\text{c}.\end{cases}
              \end{aligned}\]
    \item Рассчитывают $F=1+16\cdot(0,53-El)^3$ - коэффициент наклона.
    \item Рассчитывают задержку \[\delta\tau_{\text{ион}}=\begin{cases}F\left(A_1+A_2\cos(A_3)\right),&\text{если }|A_3|<\frac{\pi}{2};\\FA_1,&\text{если }|A_3|\geqslant\frac{\pi}{2}.\end{cases}\]
\end{enumerate}

В результате многолетних исследований установлено, что использование модели Клобушара позволяет компенсировать не менее 50\% погрешности, вызванной ионосферной рефракцией. Остаточная погрешность в этом случае составляет 10 … 15 нс, что соответствует ЭПД $\approx$ 3 … 4,5 м.

\section{Ионосферная погрешность. Оценка вертикальной ПЭС с помощью моделиNeQuick.}

Модель NeQuick (обновлённая версия называется NeQuick2) позволяет оценить вертикальную ПЭС. Исследования показали, что данная модель позволяет компенсировать не менее 70\% погрешности, поэтому она была рекомендована Международным союзом электросвязи для оценки ПЭС на основе данных о местоположении потребителя и времени, а также интегральном потоке радиоизлучения на длине волны 10,7 см. Поток радиоизлучения определяется либо числом Вольфа $R_{12}$ (самом точном из солнечных индексов), либо индексом $F_{10,7}$ (количество пятен на диске Солнца). Связь между величинами $R_{12}$ и $F_{10,7}$ описывается выражением

\[R_{12}=\frac{F_{10,7}-57}{0,93}\]

В составе НС нового навигационного радиосигнала НКА ГЛОНАСС в диапазоне L3 передаётся скорректированный индекс солнечной активности $F_{10,7}$ на дату проведения измерения, в системе Galileo эта величина называется эффективный параметр ионизации:

\[P_{\text{ион}} = a _ 0 + a _ 1 \mu + a _ 2 \mu ^ 2\]

где $a_0$, $a_1$, $a_2$ –– коэффициенты, передаваемые в составе цифровой информации радиосигнала НКА Galileo; $\mu$ – уточнённое магнитное наклонение $\mu=\frac{I}{\sqrt{\cos B}}\mu $, I –– магнитное наклонение.

\section{Ионосферная погрешность. Компенсация за счёт использования нескольких частот.}

Кроме рассмотренных моделей для компенсации ионосферной задержки могут использоваться данные, передаваемые широкозонными функциональными дополнениями.

Наиболее радикальным методом компенсации ионосферной погрешности является её оценка и компенсация за счёт использования ИТНП на нескольких частотах. Данный метод основан на том, что из всех видов погрешностей псевдозадержки, только ионосферная составляющая зависит от частоты сигнала. Поэтому разность значений псевдозадержки $\tau_1$ и $\tau_2$, измеренных на частотах $f_1$ и $f_2$ будет равна разности ионосферных задержек:

\[\Delta\tau_{\text{ион}} = \tau _ { 1 }-\tau_{2}=\delta\tau_{\text{ион}1}-\delta\tau_{\text{ион}2}\]

$\delta\tau_{\text{ионі}}$ - ионосферная задержка на частоте $f_i$.

В то же время известна теоретическая модель зависимости величины ионосферной задержки от разности частот

\[\Delta\tau_{\text{ион}} = \delta \tau_{\text{ион}1}\frac{f_{2}^{2}}{f_{2}^{2}-f_{1}^{2}}\]

с помощью которой, при известных значениях частот $f_1$ и $f_2$, может быть рассчитано значение ионосферной задержки:

\[\delta\tau_{\text{ион}1}=\frac{f_{2}^{2}}{f_{1}^{2}-f_{2}^{2}}\left(\tau_{1}-\tau_{2}\right)\]

которое используется как поправка к измерениям псевдозадержки на частоте $f_1$.

Величину ионосферной задержки на второй частоте можно получить с помощью
масштабного коэффициента $(f_1/f_2)^2$. Например, при использовании навигационных
радиосигналов ГЛОНАСС с частотным разделением масштабный коэффициент $(9/7)^2$, а
при использовании радиосигналов GPS в тех же диапазонах L1 и L2 масштабный
коэффициент составляет $(77/60)^2$.

Из выражения следуют формулы так называемых безионосферных комбинаций
измерений псевдодальности Sбезион и псевдофазы Фбезион:

\[\begin{aligned}S_{\text{безион}} & = \frac{f_{1}^{2}}{f_{1}^{2}-f_{2}^{2}}\Big[S_{1}-\frac{f_{2}}{f_{1}}S_{2}\Big]\\\\\Phi_{\text{безион}} & = \frac{f_{1}^{2}}{f_{1}^{2}-f_{2}^{2}}\Big[\Phi_{1}-\frac{f_{2}}{f_{1}}\Phi_{2}\Big]\end{aligned}\]

где Si –– измерение псевдодальности на частоте fi; Фi –– измерение псевдофазы на частоте fi.

Как следует из литературных источников, с помощью описанной коррекции удаётся обеспечить остаточную (нескомпенсированную) ионосферную погрешность на уровне порядка нескольких сантиметров (СКО).

\section{Принципы дифференциальной коррекции. Классификация методов.}

В основе дифференциальной навигации лежит свойство пространственно-временной коррелированности погрешностей измерений радионавигационных параметров сигналов НС, проведенных в различных точках пространства в близкие моменты времени.

Суть дифференциальной навигации заключается в измерении и компенсации коррелированных постоянных и медленно меняющихся во времени и пространстве составляющих погрешностей измерений навигационных параметров, т.е. измерений псевдо задержки дальномерного кода, псевдо доплеровской частоты и псевдо фазы принимаемых сигналов НС.

Задача повышения точности решается путем формирования на базовой станции (ККС) КИ – поправок, которые по соотв. каналу связи передаются потребителю и используются им для коррекции результатов НВО.

\

Варианты реализации (классификация методов):

\begin{enumerate}
    \item По системе координат:
          \begin{itemize}
              \item в абсолютной системе координат (диф. коррекция);
              \item в локальной системе координат опорной станции (относительные измерения).
          \end{itemize}
    \item По корректируемым параметрам:
          \begin{itemize}
              \item коррекция координат (требует общего рабочего созвездия, редко применяется);
              \item Коррекция ТНП (позволяет потребителю выбрать созвездие, широко применяется).
          \end{itemize}
    \item По используемым типам первичных измерений:
          \begin{itemize}
              \item псевдо дальностей и псевдо скоростей.
              \item псевдо дальностей и псевдо фаз (для сглаживания псевдо дальностей).
              \item фазы несущей
          \end{itemize}
          Первые два подхода - стандартный (обычный) ДР.
          Третий подход - высокоточный ДР (RTK).
    \item По зоне охвата:
          \begin{itemize}
              \item Локальные (используют 1 опорную станцию. Дальность до 300км)
              \item Региональные (сеть опорных станций. Дальность до 2000 км) Формируют скалярные поправки для каждого НКА отдельно. Обеспечивают более однородное качество нав. обеспечения, задержка формирования поправок.
              \item Широкозонные (сеть опорных станций + центр . Дальность до 5000км и более) Формируют вектор поправок для отдельных составляющих погрешности измерений (трехмерные поправки к эфемеридам НКА, поправки к БШВ НКА, параметры модели ионосферной задержки) Оценивают сдвиги шкал времени каждой опорной станции сети относительно опорной шкалы времени.
          \end{itemize}
\end{enumerate}

\section{Принципы дифференциальной коррекции. Структура локальной системы ДК.}

\pic{H}{\textwidth}{12}

\[\Delta \text{НП} = \text{НП}_\text{ККС} - \text{НП}_\text{эт} \]

В случае метода коррекции ИТНП для каждого НКА ДПС в составе КИ обычно передает одну скалярную поправку для суммарной погрешности измерения псевдодальности. При методе коррекции координат потребителя, КИ формируется путем сопоставления эталонных координат ККС с координатами, вычисленными входящей в ее состав АП.

В зависимости от места использования КИ в системе потребителей различают прямой и инверсный дифференциальные методы.

Прямой метод предусматривает использование КИ непосредственно в АП для уточнения параметров его вектора состояния.

Инверсный метод используется в тех случаях, когда требуется уточнение вектора состояния объекта не в точке его нахождения, а в некотором другом пункте, на котором осуществляется контроль положения объектом. При этом с АП контролируемого объекта на пункт управления транслируются результаты первичных измерений, либо текущие координаты, по которому они вычислены. В пункте управления полученные данные корректируются с использованием КИ от ККС.

\section{Методы относительных измерений. Интерферометрический метод определения относительных координат.}

Методы относительных измерений используют если абсолютные координаты объектов не представляют интереса, либо необходимая точность их измерения не обеспечивается стандартными методами.

Пусть два объекта по одним и тем же созвездиям НКА в совпадающие моменты времени определяют свои геоцентрические координаты и вычисленные координаты по соответствующему каналу связи передаются с одного объекта на другой. На каждом объекте вычисляют разности одноименных координат:

\[\Delta x=x_1-x_2\quad\Delta y=y_1-y_2\quad\Delta z=z_1-z_2\]

т. е. проекции базовой линии, соединяющей эти два объекта, на соответствующие оси координат. Затем определяют длину базовой линии (расстояние между объектами):

\[D=\sqrt{\Delta x^{2}+\Delta y^{2}+\Delta z^{2}},\]

а также углы характеризующие положение базовой линии в пространстве:

\[\alpha=\arccos\frac{\Delta x}D;\quad\beta=\arccos\frac{\Delta y}D;\quad\gamma=\arccos\frac{\Delta z}D.\]

Совокупность вычисленных параметров однозначно описывает взаимное положение объектов в пространстве.

Современные системы ОИ работают по принципу совместной обработки не координат, а навигационнных параметров объектов (псевдодальностей, псевдофаз, а также псевдодоплеровской частоты).

\

\textbf{Интерферометрический метод определения относительных координат}

\pic{H}{\textwidth}{9}

Расстояние от П и ОС до НКА:20000 км>>d, то фронт приходящей волны можно считать плоским и линии, соединяющие приемные антенны с источником – параллельными. Информация о положении П относительно БС заложена в разности хода сигналов от НКА, принятых в один и тот же момент времени.

\

Разность отсчетов псевдодальностей:
\[\begin{aligned}
         & \Delta S_{\text{ПС}} = S _ {\text{П}} - S _ {\text{БС}} = R _ {\text{П}} + \delta R +                                                                                                              \\
         & + c \cdot ( \delta t _ {\text{ШВП,П}} - \delta T _ {\text{БШВ}} + \delta \tau _ {\text{ион, П}} + \delta \tau _ {\text{троп,П}} + \delta \tau _ {\text{мнп,П}} + \delta \tau _ {\text{пром,П}} ) + \\
         & + e _ { \mathrm{S,\Pi}} - R _\text{БС}{ - \delta }R-c\cdot(\delta t_{\text{ШВП,БС}} - \delta T _ {\text{БШВ}} + \delta \tau _ {\text{ион,БС}} + \delta \tau _ {\text{троп,БС}} +                   \\
         & +  \delta \tau _ {\text{мнл,БС}} + \delta \tau _ {\text{прм,БС }} ) - e _ { \mathrm{S,BC}}
    \end{aligned}\]

Разность отсчетов псевдофаз:

\[\Delta\Phi_{\text{ ПС }}=\Phi_{\text{ П}}-\Phi_{\text{ БС}}=\frac{R_{\text{П}}+\delta R}{\lambda}+\]
\[+f\cdot(\delta t_{\text{ШВП,П}}-\delta T_{\text{БШВ}}-\delta\tau_{\text{ионД }}+\delta\tau_{\text{ троп,П}}+\delta\tau_{\text{мнл,П}}+\delta\tau_{\text{прм, П}})+N_{\text{П}}+e_{\Phi.\Pi}-\]
\[-\frac{R_{\text{БС}}+\delta R}{\lambda}+f\cdot(\delta t_{\text{ШВП,БС}}-\delta T_{\text{БШВ}}-\delta\tau_{\text{иoн, БС}}+\delta\tau_{\text{троп,БС}}+\]
\[+\delta\tau_{\text{мнл,БС}}+\delta\tau_{\text{прм,БС}})+N_{\text{ БС}}+e_\text{Ф,БС}\]

Учитывая идентичность условий распространения, получим выражения для первых разностей псевдодальностей и псевдофаз:

\[\begin{aligned}\Delta S&=\Delta R+c\cdot(\Delta t+\Delta\tau_\text{ мнл }+\Delta\tau_\text{прм })+\varepsilon,\\\Delta\Phi&=\frac{\Delta R}{\lambda}+f\cdot(\Delta t+\Delta\tau_\text{мнл }+\Delta\tau_\text{прм })+N+\eta,\end{aligned}\]

\begin{tabular}{ l c l }
    $\Delta R$                & - & разность дальностей от НКА до потребителя и БС,       \\
    $\Delta t$                & - & разность расхождения шкал времени АП и БС,            \\
    $\Delta \tau_\text{мнл}$  & - & \makecell[l]{задержка радиосигналов, обусловленная    \\ многолучевым характером распространения сигналов,} \\
    $\Delta\tau_{np\text{м}}$ & - & \makecell[l]{разность задержек радиосигналов          \\ в приемных устройствах АП и БС,} \\
    $N$                       & - & разность ПФН(параметр фазовой неоднозначности),       \\
    $\varepsilon$             & - & \makecell[l]{суммарная аппаратурная                   \\ погрешность разности псевдодальностей,} \\
    $\eta $                   & - & суммарная аппаратурная ошибка измерений разности фаз. \\
\end{tabular}

\

Для сигналов с кодовым разделением (на одной несущей) можно исключить разность расхождений шкал приемников и аппаратные задержки в приемной аппаратуре сформировав вторые разности:

\[\begin{aligned}\Delta_2S_{ij}&=\Delta S_i-\Delta S_j=\Delta_2R_{ij}+\mu_{ij}+\boldsymbol{\varepsilon}_{ij},\\\Delta_2\Phi_{ij}&=\Delta\Phi_i-\Delta\Phi_j=\frac{\Delta_2R_{ij}}\lambda+N_{ij}+\nu_{ij}+\boldsymbol{n}_{ij}\\\text{где і и ј -номера Нка, }\Delta_2R_{ij}&=\Delta R_i-\Delta R_j\end{aligned}\]

При обработке сигналов ГЛОНАСС с частотным разделением сигналов, вторые разности описываются следующими выражениями:

\[\begin{aligned}\Delta_2S_{ij}&=\Delta_2R_{ij}+c\cdot(\Delta t+\Delta t_{\text{прм}} ) + \mu _ { i j }+\varepsilon_{ij},\\\Delta_2\Phi_{ij}&=\frac{\Delta R_i}{\lambda_i}-\frac{\Delta R_j}{\lambda_j}+(f_i-f_j)\cdot(\Delta t+\Delta t_{\text{прм}}) + N _ { i j }+\mathrm{v}_{ij}+\mathrm{\eta}_{ij},\end{aligned}\]

где

$\mu_{ij}$ - суммарная ошибка измерений псевдодальности, обусловленная многолучевым характером распространения, $\mu_{ij}=\mathcal{C}\left(\Delta\tau_{\textit{мнл }i}-\Delta\tau_{\text{мнл }j}\right)$

$\varepsilon_{ij}$ - суммарная шумовая и динамическая ошибка измерений псевдодальности.

Исключение слагаемого $\Delta t+\Delta t_{\text{прм}}$ возможно в предположение об идентичности и равномерности характеристик группового времени запаздывания (ГВЗ) сигналов в приемниках в полосе сигналов НКА. Тогда получим:

\[\Delta t+\Delta t_\text{прм }=\frac1M\sum_{i=1}^M\frac{\Delta S_i-\Delta R_i}c,\]

где M – количество используемых НКА.

\section{Разностные методы устранения фазовой неоднозначности. Методы оценки параметра фазовой неоднозначности.}

Определение N (раскрытие фазовой неоднозначности) состоит в построении третьих разностей, формируемых путем вычитания вторых разностей относящихся к разным моментам времени при условии непрерывного слежения за сигналами НКА в течении порядка 30мин.

Для сокращения времени раскрытия фазовой неоднозначности используют методы:

\begin{itemize}
    \item перемещение антенн при непрерывном слежении за сигналом («stop\&go», «реокупация»);
    \item избыточных измерений;
    \item априорная оценка координат;
    \item использование навигационных средств иного типа.
\end{itemize}

\begin{enumerate}
    \item Перемещение антенн:в процессе проведения измерений, антенны приемника и опорной станции переносят так, чтобы поменять местами точки их установки.\\
          Значения ПФН получают линейной комбинацией первичных измерений.\\
          Алгоритм «stop\&go»:Используют то, что ПФН не изменяется при взаимном движении антенн.
    \item Метод «реокупации»(возвращенияна исходную точку):в течение навигационного сеанса НАПсовершает накопление первичных измерений в исходной точке, перемещение из опорной точки к точкам, чьи относительные координаты необходимо определить, и обратно.
    \item Метод априорной оценки относительных координат:позволяет определить неоднозначность по одному отсчету. Эффективен при слежении за деформациями сложных конструкций или перемещениями тектонических плит земной коры. Если априорная информация недостаточна точна, то для разрешения неоднозначности используется ряд методов, например, Ambiguity Function Method (AFM). Суть метода в поиске максимума функции неоднозначности вида: \[AF(\vec{X})=\sum_{k=1}^{m}\sum_{l=1}^{n}\left|\sum_{j=1}^{m_{kl}}e^{i\cdot\left(\Delta\Phi_{ij}(t_k)-\Delta R_{jl}(t_k,\vec{X})/\lambda_{ij}\right)}\right|\] 
    
    Где:

    $\vec{X}$ - искомый вектор относительных координат

    $\Delta\Phi_{ij}(t_k)$ - первая разность измерений фаз сигналов j-го НКА на частоте i потребителем и БС в момент времени $t_k$

    $\Delta R_{jl}(t_k,\vec{X})$ - первая разность расчетных значений псевдодальности разностей от j-го НКА до П и БС

    $\lambda_{ij}$ - длина волны несущих колебаний сигнала j-го НКА на частоте i

    $m_{kl}$ - количество первых разностей

    $m$ - количество отсчетов (моменты времени)

    $n$ - количество несущих частот, на которых производятся измерения

    Различным значениям вектора X соответствуют разные значения функции неоднозначности. Целью процедуры поиска является определения вектора, соответствующего максимуму AF(X).
    \item Использование избыточных измерений (RTK): наиболее распространен. Используя измерения по всем возможным НКА вычисляют вещественную оценку ПФН, а затем переходят к целочисленной оценке.
\end{enumerate}



\section{Дополнительные источники погрешностей при высокоточных измерениях}

Методы РРР базируются на учете при НВО ряда погрешностей измерений текущих навигационных параметров (ИТНП), которыми в режиме абсолютных определений со стандартной точностью обычно пренебрегают в связи с их малостью; при высокоточном определении местоположения такие пренебрежения становятся недопустимыми. Такие погрешности обусловлены наличием в измерениях систематических ошибок (смещений), к которым относятся:

\begin{itemize}
    \item релятивистские смещения;
    \item гравитационные смещения;
    \item смещения вызываемые приливами в упругом теле Земли, полярными, океаническими и атмосферными приливами, движением тектонических плит Земли;
    \item смещения в псевдофазовых измерениях, связанные с взаимной ориентацией антенн НКА и АП;
    \item смещения (а также случайные вариации) положения фазовых центров антенн НКА относительно центров масс спутников;
    \item смещения ( а также вариации) положения фазовых центров антенн АП относительно геодезических координат точек их установки.
\end{itemize}

\section{Расширенные модели первичных измерений. Модель псевдодальности.}

Наличие в измерениях неопределённых составляющих, которые не могут быть оценены и скомпенсированы с помощью математических моделей, используемых при стандартных измерениях, ограничивает их точность. Для высокоточных измерений необходим переход от традиционной математической модели измерений, в которой псевдодальность S включает в себя следующие составляющие погрешности:

\begin{itemize}
    \item эфемеридную погрешность ${\delta}{R}$
    \item расхождения БШВ и ШВП относительно СШВ
    \item задержку распространения радиосигнала в ионосфере
    \item задержку распространения радиосигнала в тропосфере
    \item задержку распространения радио- сигнала вследствие многолучевости
    \item задержку радиосигнала в АП
\end{itemize}

\[S=R+\delta R+c\cdot\left[\delta t_{\text{ШBП}}-\delta T_{\text{БШB}}+\delta\tau_{\text{HKA}}+\delta\tau_{\text{peл}}+\delta\tau_{\text{ион}}\right.+\]
        \[\left.+\delta\tau_{\text{троп}}+\delta\tau_{\text{мнл}}+\delta\tau_{\text{прм}}\right]+\delta R_{\mathrm{\Gamma.\Pi}}+\delta R_{\mathrm{A}}+e_{S},\]

где:

$\delta\tau_{\text{peл}}$ - величина релятивистской поправки;

$\delta\tau_{\text{HKA}}$ - величина групповой задержки радиосигнала в бортовом источнике навигационного сигнала НКА;

$\delta R_{\mathrm{\Gamma.\Pi}}$ - величина смещения ИТНП, обусловленное гравитационно-приливными эффектами;

$\delta R_{\mathrm{A}}$ - смещение фазового центра передающей антенны НКА относительно центра масс НКА.



\section{Расширенные модели первичных измерений. Модель псевдофазы.}

\[\lambda\Phi=R+\delta R+c\cdot[\delta t_{\text{ШВП}}-\delta T_{\text{БШB}}+\delta\tau_{\text{peл}}-\delta\tau_{\text{ион}}+\]
        \[+\delta\tau_{\text{троп}}]+\delta R_{\text{г.п.}}+\delta R_{\text{A}}+\lambda[\delta\psi_{\text{HKA}}+\delta\varphi+\delta\Phi_{\text{BO}}+\]
        \[+\delta\psi_{\text{мнл}}+\varphi_{0}-\psi_{0}+N]+e_{\Phi}\]

Где:

$\delta\psi_{\text{HKA}}$ - фазовая задержка несущих колебаний радиосигнала в бортовом источнике навигационного сигнала НКА;

$\delta\varphi$ - фазовая задержка несущих колебаний в АП;

$\delta\Phi_{\text{BO}}$ - смещение, порождаемое изменениями взаимной ориентации антенн АП и НКА;

$\delta\psi_{\text{мнл}}$ - фазовые искажения принимаемого сигнала при многолучевом распространении (переотражениях);

$\psi_{0}$ - начальная фаза несущих колебаний радиосигнала НКА;

$\varphi_{0}$ - начальная фаза колебаний на несущей частоте, формируемых в АП.

\section{Технология высокоточного эфемеридно-временного обеспечения. Международные службы обработки и обеспечения ЭВО. Классификация типов данных ЭВО.}

Роль пионера в исследовании технологий высокоточного ЭВО принадлежит организациям, входящим в IGS (IGS -- International GNSS Service), которая первоначально называлась Международная служба GPS. Для того чтобы лучше понимать потенциально достижимые точности определения параметров движения и БШВ КА рассмотрим более подробно выходные данные, которые формируются на основе самой развитой сети измерительных станций в мире. Собранные данные используются для формирования выходных данных (продуктов) IGS, которые подразделяются на следующие четыре группы.

\begin{enumerate}
    \item Финальные продукты (final products), включающие эфемериды и ЧВП НКА ГЛОНАСС и GPS, параметры вращения Земли (движение полюса и длительность суток), параметры шкал времени, координаты и параметры вектора скорости станций. Финальные продукты определяются на временном интервале, соответствующем неделе GPS, и предоставляют не позднее априорно определённого срока (в настоящее время через 10 суток с момента окончания временного интервала).
    \item Быстрые продукты (rapid products) включают ЭВИ только для НКА GPS, а также оценки параметров вращения Земли и параметры шкалы времени станций на суточном интервала и формируются в течении 17 ч после окончания соответствующих суток.
    \item Сверх-быстрые продукты (ultra-rapid products) –– эфемериды и ЧВП НКА GPS, эфемериды НКА ГЛОНАСС и оценки параметров вращения Земли, вычисляемые 4 раза в сутки в течении 3 ч после формирования последнего ИТНП. Данные продукты включают апостериорную оценку ЭВИ за последние 24 ч, а также прогноз ЭВИ также на интервале 24 ч.
    \item Продукты реального времени (real-time products) включают КИ к бортовой ЭВИ НКА ГЛОНАСС, а также потоки ИТНП с глобальной сети ССИ. В рамках экспериментальных исследований IGS проводит испытания услуги формирования КИ к ЭВИ НКА ГЛОНАСС. Эфемериды в данном режиме обновляются с периодичностью не более 60 с, а ЧВП – 5 с.
\end{enumerate}

Анализ погрешностей продуктов IGS показывает, что развитие методов оценки и прогнозирования ЭВИ позволяет уменьшить значения $\delta R$ и $\delta T_\text{БШВ}$ до долей метра, что существенно меньше большинства составляющих погрешности ИТНП, рассмотренных и описанных в моделях.

\section{Релятивистские поправки БШВ.}

Уже отмечалось, что на показания БШВ НКА влияют релятивистские эффекты, которые следуют из специальной и общей теорий относительности: орбитальное движение НКА приводит к замедлению бортового времени, уменьшение гравитационного потенциала на высоте орбиты – к ускорению. При скорости движения НКА порядка 4 км/с и высоте орбиты порядка 20 000 км суммарный эффект указанных факторов приводит к ускорению хода бортового времени относительно земного. Соответственно, с точки зрения потребителя, находящегося на Земле, опорная частота БСУ НКА будет больше, чем ее номинальное значение на борту НКА. Для того чтобы скомпенсировать этот эффект, частоты задающих генераторов БСУ НКА ГЛОНАСС снижают на относительную величину $\Delta f/f=-4,36\cdot10^{-10}$.

В интерфейсном контрольном документе (ИКД) GPS описана специальная процедура коррекции показаний БШВ НКА, основанная на расчете величины $\delta \tau_\text{рел}$ по параметрами орбиты НКА. Однако, более точным является метод мгновенного вычисления $\delta \tau_\text{рел}$:

\[\delta\tau_{\text{pe}\text{л}} = - \frac { 2 \boldsymbol{P}\cdot\boldsymbol{V}}{2}\] где P – вектор координат НКА; V – вектор скорости НКА.

Входящие в это выражение векторы могут быть заданы как в подвижной гринвичской системе координат, связанной с вращающейся Землёй, так и в инерциальной системе координат, связанной с неподвижной Землёй, следовательно, оно может использоваться IGS при расчете параметров моделей движения и шкал времени НКА GPS и ГЛОНАСС в любой из этих систем.

Расчёты по этой формуле показывают, что значение $\delta\tau_{\text{pe}\text{л}}$ может достигать 15.. .20 нс, что соответствует поправке в ИТНП порядка 4,5…6м. Очевидно, что поправки такого уровня должны обязательно учитываться при высокоточном определении координат потребителей.

\section{Гравитационно-приливные эффекты. Величина смещения ИТНП,обусловленная гравитационными эффектами.}

Для учета изменения формы Земли и ее гравитационного поля под действием других небесных тел, а также полярных и океанических приливов, вводится величина $\delta R_\text{г.п.}$ определяемая следующим выражением:

\[\delta R_\text{г.п.}=\delta R_\text{г}+\delta R_\text{т}+\delta R_\text{0}+\delta R_\text{п},\]

где:

$\delta R_\text{г}$ - величина смещения ИТНП, обусловленная гравитационными эффектами;

$\delta R_\text{т}$ - величина смещения ИТНП, обусловленная приливами в упругом теле Земли;

$\delta R_\text{0}$ - величина смещения ИТНП, обусловленная океаническими приливами;

$\delta R_\text{п}$ - величина смещения ИТНП, обусловленная полярными приливами.

Смещение $\delta R_\text{г}$ определяется влиянием гравитационных полей Земли, Луны, Солнца, Юпитера и других небесных тел и может быть рассчитано как сумма смещений, вызванных каждым из них. Для оценки смещения, вызванного гравитационным полем одного небесного тела, используется формула

\[\delta R_\text{г}^{\mathrm{HT}}=-R_s\ln(1-P*X),\]

где:

$R_s$ - радиус Шварцшильда $R_S=2GM/{c}^2$

$G$ - гравитационная постоянная;

$M$ – масса небесного тела;

$\boldsymbol{P}$ - единичный вектор, направленный от АП к НКА;

$\boldsymbol{X}$ - единичный вектор, направленный от АП к данному небесному телу.

На практике оказывается, что ближайшая к Земле планета-гигант Юпитер вносит гравитационную задержку порядка нескольких сотен пикосекунд, т. е. вклад в гравитационную задержку составляет менее 0,01 м. Наибольшее вклад вносит смещение гравитационного поля Земли, соответствующая задержка может достигать нескольких сантиметров. Очевидно, что эту величину необходимо учитывать, поскольку она сопоставима с величиной аппаратурной погрешности ИТНП.

\section{Гравитационно-приливные эффекты. Величина смещения ИТНП,обусловленная приливами в упругом теле Земли.}

Смещение обусловлено тем, что Земля не является абсолютно твёрдым телом – под действием внешних сил расстояние между двумя произвольными точками изменяется. Она не является и абсолютно упругим телом. Если действие внешних сил прекращается, то точки не возвращаются в первоначальное положение, т. е. Земля остаётся в деформированном состоянии.

Для описания упругих деформаций Земли английский геофизик А.Э. Х. Ляв (Лав) ввёл безразмерные параметры h (позже японский учёный Шида определил число l), которые сейчас называются числами Лява и Шида. Числа Лява связаны с модулями упругости Земли (величинами, характеризующими упругие свойства материалов при малых деформациях). В настоящее время доказано, что упругие свойства Земли, и, следовательно, числа Лява зависят от частоты воздействующей на Землю силы.

Потенциал в точке наблюдения складывается из гравитационного потенциала внешних тел (Солнца, Луны и планет) и собственного потенциала Земли, возмущённого приливными деформациями. Для определения положения пункта в системе ITRF из его мгновенного радиус вектора вычитается вектор полной приливной деформации, причём постоянное смещение вычисляется с использованием чисел Лява и Шида.
Для принятой модели строения Земли были рассчитаны числа Лява, на основе которых вычисляется приливное смещение АП с погрешностью менее 0,005 м (СКО) по формуле

\[\delta R_T = \sum_{j=2}^{3}\frac{GM_{j}}{GM}\frac{r^{4}}{R_{j}^{3}}\Big\{\Big[3l_{2}(\boldsymbol{R}_{j}\cdot\boldsymbol{r})\Big]\cdot\boldsymbol{R}_{j}+\Big[3\cdot\Big(\frac{h_{2}}{2}-l_{2}\Big)(\boldsymbol{R}_{j}\cdot\boldsymbol{r})^{2}-\frac{h_{2}}{2}\Big]\cdot\boldsymbol{r}\Big\}+\]
\[+\left[-0{,}025m\sin\varphi\cos\varphi\sin(\theta_g+\lambda)\right]\cdot\boldsymbol{r},\]

где:

$M_j$ - масса Луны (j = 2) и Солнца (j = 3);

$r, R_j$ - модули геоцентрических векторов АП, Луны и Солнца и соответствующие им единичные вектора;

$I_2$ - число Шида 2-го порядка (0,609); 

$h_2$ - число Лява 2-го порядка (0,085); 

$\phi$ - широта пункта;

$\lambda$ - долгота пункта;

$\theta_g$ - среднее Гринвичское сидерическое время.

Значение поправки может достигать 0,3 м в радиальном направлении и 0,05 м в плане. Поскольку внешний потенциал включает как зависящие от времени гармоники, так и постоянную во времени часть, то и приливное смещение АП содержит постоянную и переменную во времени компоненты. Постоянная компонента смещения в средних широтах может достигать 12 см и оставаться неизменной в течение 24 ч, поэтому расчёт поправок, учитывающих данный вид погрешностей, должен обязательно входить в алгоритмы высокоточного местоопределения.

\section{Гравитационно-приливные эффекты. Величина смещения ИТНП,обусловленная океаническими приливами.}

Кроме описанных выше эффектов, при оценке координат необходимо также учитывать влияние приливов мирового океана $ \delta R_o $ . В основе данного эффекта лежит гравитационно-приливное взаимодействие в системе Земля - Луна, которое приводит к изменению орбиты Луны и замедление вращения Земли.

Масса Луны сравнительно велика (1/81 массы Земли) и она расположена довольно близко, вызывая приливы на Земле. В океанских водах на обращённой к Луне стороне формируется приливная волна (такая же волна формируется и на противоположной стороне).
Если бы Земля не вращалась вокруг своей оси, приливная волна находилась бы точно под Луной, которая притягивает её к себе, и бежала бы по поверхности Земли с запада на восток, совершая полный оборот за один сидерический лунный месяц (27 дней 7 ч 43,2 мин).

Однако Земля вращается "под" этой волной, совершая один оборот за сидерический день (23 ч, 56 мин, 4,091 с).
В результате приливная волна бежит по поверхности Земли с востока на запад, совершая один полный оборот за 24 ч 48 мин.
Поскольку Земля не является идеальным шаром, на каждую из этих двух приливных волн регулярно "набегают" восточные берега материков, омываемых Мировым океаном ("набегают" именно материки на волну, поскольку Земля вращается быстрее обращения Луны). 
Из-за этого приливная волна смещается вперёд по направлению вращения Земли, опережая Луну. Следствием такого опережения является то, что значительная часть массы океанских вод (т. е. и часть массы всей Земли) смещается вперёд относительно линии, соединяющей центры масс Земли и Луны.
Эта смещённая вперёд масса притягивает к себе Луну, создавая силу, действующую перпендикулярно линии Земля –– Луна.
В результате на Луну действует момент силы, ускоряющий её обращение по орбите вокруг Земли.

\pic{H}{\textwidth}{11}

Международная служба вращения Земли IERS разработала рекомендации, согласно которым смещение ИТНП вследствие влияния приливов в каждом направлении можно оценить по следующей формуле:

\[\delta R_o =\sqrt{\sum_{i=1}^{3}\left[\sum_{j=1}^{11}f_{j}A_{ij}\cos(\omega_{j}t+\chi_{j}+u_{j}-\Phi_{ij})\right]^{2}}\cos\alpha,\]

где:

i - индекс, обозначающий направления на центр Земли (i = 1), на запад (i = 2) и на юг (i = 3); 

j - индекс, обозначающий суммирование по 11 приливным волнам (полусуточные волны $M_2$, $S_2$, $N_2$ и $K_2$; суточные волны $K_1$, $O_1$, $P_1$ и $Q_1$; длительные волны $M_f$ , $M_m$ и $S_sa$);
$f_j, u_j$ - коэффициенты, зависящие от долготы лунного узла (при точности в 1...3 мм $f_j$ = 1 и $u_j$ = 0); 

$\omega_{j}$ - угловая скорость вращения Земли; 

$\chi_{j}$ - астрономический аргумент; 

$A_{ij}, \Phi_{ij}$ - амплитуда и фаза смещения (по радиусу, на запад и на юг) местоположения потребителя; 

$\alpha$ - угол между векторами смещения и направления на НКА из точки расположения потребителя.

Величины $\omega_{j}$ и $\chi_{j}$ принимают равными значениям угловой скорости вращения Земли и астрономического аргумента в полночь.
Значения амплитуды $A_{ij}$ и фазы $\Phi_{ij}$ рассчитывают с использованием модели движения мирового океана и данных береговых наблюдений, которые можно найти, например, на сайте национального управления океанических и атмосферных исследований (NOAA) или в национальных гидрографических службах. 
Результаты многолетних наблюдений свидетельствуют, что амплитуда изменения момента M2 является самой большой и не превышает для прибрежных станций 0,05 м по радиусу и 0,02 м в горизонтальном направлении. 
При решении задач высокоточного определения местоположения необходимо учитывать не только упругие деформации Земли, но и влияние приливов. 
Причём делать это следует не только для потребителей находящихся в прибрежной зоне, но и для тех из них, кто расположен в глубине материков.

\section{Гравитационно-приливные эффекты. Величина смещения ИТНП,обусловленная полярными приливами.}

Под полярными приливами понимают периодические смещения оси вращения Земли по отношению к ее поверхности. 
Данные смещения вызваны изменениями гравитационного потенциала Земли под воздействием Солнца и Луны. 
Смещение ИТНП $\delta R_\text{п}$ может быть вычислено на основании данных о смещении АП по широте $\Delta B$, долготе $\Delta L$ и высоте $\Delta H$ по следующим формулам:

\[\delta R_{\mathrm{n}}=\sqrt{\Delta X^{2}+\Delta Y^{2}+\Delta Z^{2}}\cos\alpha;\] \\
\[[\Delta X,\Delta Y,\Delta Z]^{\intercal}=Q^{\intercal}[\Delta B,\Delta L,\Delta H]^{\intercal};\] \\
\[\Delta B=-9\cos(2B)\big[\big(X_{\mathrm{n}}-\overline{X}_{\mathrm{n}}\big)\cos L-\big(Y_{\mathrm{n}}-\overline{Y}_{\mathrm{n}}\big)\sin L\big]; \] \\
\[\Delta L=9\sin(2B)\big[\big(X_\mathrm{n}-\overline{X}_\mathrm{n}\big)\sin L-\big(Y_\mathrm{n}-\overline{Y}_\mathrm{n}\big)\cos L\big]; \] \\
\[\Delta H=-33\sin(2B)\big[\big(X_{\mathrm{n}}-\overline{X}_{\mathrm{n}}\big)\cos L-\big(Y_{\mathrm{n}}-\overline{Y}_{\mathrm{n}}\big)\sin L\big], \]

где:

$\Delta X, \Delta Y, \Delta Z$ - смещение координат АП;
$\alpha$ - угол между векторами смещения и направления на НКА из точки расположения потребителя; 
$B, L$ - координаты АП;
$X_\text{п}, Y_\text{п}$ - координаты полюса;
$\overline{X}_\text{п}, \overline{Y}_\text{п}$ - средние координаты полюса.

В отличие от твердотельных приливов, смещения полюсов Земли не усредняются на суточном интервале времени. 
Они постепенно изменяются с периодом около 430 дней. 
Максимальные значения коррекций, связанные со смещениями полюсов Земли, достигают 0,0025м по высоте и 0,007м в горизонтальной плоскости.


\section{Аппаратурные задержки.}
\section{Ионосферная погрешность.}
\section{Тропосферная погрешность. Модель Нейла.}
\section{Уточнённые модели первичных измерений.Расширенный вектор состояния потребителя.}
\section{Разрешение неоднозначности псевдофазовых измерений (Float и Integer методы).}
\end{document}